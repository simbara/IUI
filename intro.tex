\section{Introduction}
When are people available to work or communicate? Ubiquitous computing platforms like the mobile phone have become the gateway through which people access information and interact with others, in social and even physically engaged situations. The trend in technology can now  flip the paradigm and proactively provide information to the user. People, however, have a finite mental resource and can only process a limited amount of information without degradation of task performance. For cognitively challenging activities like driving, or even timing sensitive conversations like auctions or playing cards for money, divided attention can have dire consequences. Thus, there is a need for systems to gauge the load on this mental resource in order to predict or preempt degradation in task performance, while interacting with a user.

While progress has been made towards gauging this load, we are still a long way off from being able measure it in anything like a real time fine-grained level~\ref{todo}. In the future such capabilities might reduce task load to avert human mistakes. For instance, as speech technologies get better, voice interaction might be the most efficient way to interact with the user, when the user's manual and visual resources are occupied. A rapid and fine-grained measure would allow a dialog or proactive agent to track in real time the ebbs and flows of the load being experienced by the user. This would allow it to preempt disfluencies and other irregularities in speech, as well as to time its responses and other actions, so as to prevent overloading the user. In the driving scenario, its been shown that passengers adapt their conversation to task load, which leaves the driver with more resources to perform the driving task when it gets difficult~\cite{drews2008, cohen2014}. Interactive agents should aim to emulate such considerate behaviors.

This problem might be addressed by directly modelling the driver via psychophysiological measures, or by modelling driving context and its effect on the driver, or by jointly modelling both. Compared to modelling the driving context, less progress has been made in modelling the driver in order to identify when to interrupt them, or in factoring in the load from multitasking~\cite{kim2015}. One advantage of pursuing such an approach of modelling the user's internal state is the potential for these models to generalize to other domains, apart from driving. On the other hand, methods based on modelling situational context require specific sensors and modelling techniques to be considered for each domain separately. Furthermore, a psychophysiology-based approach can be tuned for each user individually, as different users might experience varying loads in a given situation. While costs and intrusiveness have stymied their widespread adoption, recent advances in wearable technologies suggest that monitoring at least a few physiological signals in everyday life might become a feasible option.

In this paper, we evaluate several signals that might be used in a psychophysiological approach to gauging user load while engaged in high task load and multitasking experiments. We wanted to capture the temporal aspects of divided attention --- the transitions in load when addressing and recovering from interruptions, for example. At the same time, we wanted to facilitate data collection that was repeatable with performance measures of a resolution suitable for real time task load evaluation or mediation. To meet these goals, we designed a driving-like primary task, with an intermittent secondary task of attending to and responding to notifications. As part of our initial explorations, we present results from two separate user studies. First, we created models that can detect which tasks the user is engaged in based on measurements of pupil dilation. We show how the performance of the model varies by changing the modality and timing of the notifications. We do this for each user, as well as across all users. Second, we built a real-time load detector based on pupil dilation measures, and used it to mediate notifications as part of a second user study. To the best of our knowledge, this is the first demonstration that pupil size variation can recognize and mark real time changes in task load. We present evaluations of the performance of the real time load detector, and the effect of its mediation on user task performance. 