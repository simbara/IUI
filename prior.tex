\section{Related Work}
In cognitive psychology, there is a general consensus that people have limited and measurable cognitive capacities for performing mental tasks~\cite{miller1956}. Furthermore, engaging in one mental task interferes with the ability to engage in other tasks, and can result in reduced performance on some or all of the tasks as a consequence~\cite{kahneman1973}. To characterize the demand on these limited resources, psychologists have employed notions like cognitive load and mental workload, which gains definition through the experimental methods that are used to measure it \cite{klingner2010}. 

\subsection{Measuring Cognitive Load}
Cognitive load can be assessed using data gathered from three empirical methods: subjective data using rating scales, performance data using primary and secondary task techniques, and psychophysiological data from sensors~\cite{paas2003}. Self-ratings, being post-hoc and subjective in nature, make them inacurate and impractical to use when automated and immediate assessment is required. Secondary task techniques are based on the assumption that performance on a secondary measure reflects the level of cognitive load imposed by a primary task. A secondary task can be as simple as detecting a visual or auditory signal, and can be measured in terms of reaction time, accuracy, and error rate. However, in contexts where the secondary task interferes with the primary task, physiological proxies that can measure gross reaction to task load are needed to assess cognitive load.

\subsubsection{Psychophysiological Measures}
Psychophysiological techniques are based on the assumption that changes in cognitive functioning cause physiological changes. An increase in cortical activity causes a brief, small autonomic nervous response, which is reflected in signals such as heart rate (HR) and heart rate variability (HRV)~\cite{fredericks2005, mulder1992, wilson2002}, electroencephalogram (EEG)~\cite{ryu2005, wilson2002}, electrocardiogram (ECG)~\cite{ryu2005}, electrodermal activity (EDA)~\cite{ikehara2005, shi2007}, respiration~\cite{mulder1992}, and heat flux~\cite{haapalainen2010},eye movements and blink interval~\cite{beatty2000, ikehara2005, iqbal2005, wilson2002} pupillary dilations. Our dataset includes most of these signals as well as additional signals that have been shown to be sensitive to affect like pulse transit time (PTT), facial electromyography (EMG) and skin temperature~\cite{liu2008, or2007}. 

In this work strivesto correlate the effects of several physiological measures of cognitive function simultaneously in a multisensory motor scenario . In particular, brain activity as measured through event-related potentials using EEG, or as inferred from pupillary responses have received more attention recently because of their high sensitivity and low latency~\cite{antonenko2010, marshall2007, klingner2010}. There has been very little work that correlates these measures with the other physiological measures, or demonstrates how to effectively align them. Furthermore, to the best of our knowledge this is the only work that has focused on tracking cognitive load that is rapidly and randomly changing, since we are interested in teasing out the dynamic nature of instantaneous cognitive load.  Lastly, typical work has focuses on cognitive load arising in single-task scenarios like document editing~\cite{iqbal2005}, and traffic control management~\cite{shi2007}. In contrast, we employ a multitasking scenario, aspects of which we briefly review below.

%Leaving out the rest periods, which is usually employed between low and high control conditions allows us to capture the temporal patterns in these signals as they transition from low to high workloads, and vice versa.

\subsection{Multitasking Scenarios}
In multitasking scenarios, the distribution of cognitive resources when engaged in two or more tasks is not very well understood. This makes it difficult to assess and predict workload that will be experienced by the user. Theories have been proposed to model how multiple tasks might compete for the same information processing resources~\cite{wickens2008, baddeley2003}. One widely used approach that has been shown to fit data from multitask studies is Wicken's multiple resource theory. This attempts to characterize the potential interference between multiple tasks in terms of dimensions of stages (perceptual and cognitive vs. spatial), sensory modalities (visual vs. auditory), codes (visual vs. spatial), and visual channels (focal vs. ambient)~\cite{wickens2008}. Performance will deteriorate when demand for one or more tasks along a particular dimension exceeds capacity. 

In the case of driving and notification comprehension, both tasks compete for resources along the stages dimension. We would expect performance to deteriorate when there is an increased demand for the shared perceptual resources, i.e. when driving is hard and/or when the notification is difficult to comprehend. If the notification is visual, both tasks might also compete along the modality and visual channel dimensions. We would expect performance deterioration to be greater for visual notifications. 

\subsubsection{Driving and Language}
Listening and responding to another person while driving a car has been widely studied, and has been shown to effect driving performance, particularly with remote conversants~\cite{kubose2006}. Passengers sitting next to a driver are able to adapt their conversation to the traffic situation, allowing the driver to focus on driving when it becomes difficult~\cite{drews2008, cohen2014}. These findings have motivated research towards building dialog systems that are situation-aware and interrupts itself when required~\cite{kousidis2014}. As mentioned earlier, the focus in most of this work is on monitoring the driving environment, and less on determining the cognitive load of the driver. Recently, there has been an interest in studying the effect that complex linguistic processing can have on driving using physiological measures of pupil dilation and skin conductance~\cite{demberg2013}. 

