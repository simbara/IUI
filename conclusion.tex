\section{Discussion}
This study compiles a dataset of psychophysiological signals to estimate cognitive load.  The measurments included  Electrodermal Activity,MechanoMygram Electro, Skin temperatures, ElectroCardioGram , Photoplethysmoragraph, ImpedenceCariograph,  Respiration, gaze, and pupil dilation. Apart from the eye-based signals, which were sampled at 30 Hz, the rest of the signals were sampled at 2000 Hz. 
Deriviative signals of Instantaneous heartrate, Pulse Transit Time and Difference in Skin temperature are also calculated in the dataset. 

Three derivative signals were also calculated. Instantaneous heart rate (IHR) was obtained from the ECG signal using the BioSig library\footnote{http://biosig.sourceforge.net/} which implements Berger's algorithm~\cite{berger1986}). Pulse Transit Time (PTT) was obtained by calculating the difference in between the ECG R-wave peak time and the PPG peak time, which is the time it takes for the pulse pressure waveform to propagate through a length of the arterial tree. Difference in skin temperature (SKT) was also Heartate variabiliy, HeartrateSituations of divided attention are now commonplace due to technological advancements. 

The Primary and secondary tasks were created to be difficult enough to elicit significant errors when a secondsary task was added for high task load.  The task load is created by a primary senimotor steering/reaction task and a secondary verbal/cognitive task to satisfy: a) a multitasking scenario that balanced the need for realistic interactions, with the need for repeatability, and b) a dataset that captured the dynamic nature of cognitive load as reflected in the participants' physiological responses. To this end, we used the CoNTR driving like scenario, where the participant intermittently attended to notifications as a secondary task. We also modulated the timing and modality of the notifications to understand their effects on cognitive load. ??? reference for verbal tasks???

We cast the multitasking scenario as a multilabel learning problem, and assigned labels to the primary and secondary tasks. We evaluated the utility of this approach, and showed that we were able to to use machine learning to build models that distinguish whether the user is engaged in the primary task, the secondary task, or both. This was demonstrated for each user individually, and across all the participants. These models were built using a number of statistical features derived from measurements of pupil dilation, which were fed to a random forest classifier. Furthermore, our evaluation showed that compared to the modality of delivery, the timing of the notifications have a larger effect on the load experienced by the user while driving. 

To show that pupil size variation metrics can be robust we used inexpensive off the shelf measuring technique. Prior work report use of expensive eye-trackers with higher sampling rates for pupilometric measurements. Other eye-camera measures of blink rate and gaze activity were not included in this analysis. Our study succeeded with an off-the-shelf webcam (Microsoft LifeCam HD-6000) with a sampling rate of 30 Hz specially mounted to view the pupil.  Even when tested outdoors in a car during the day,  with no special attempt to control for ambient luminescence the system was effective at producing pupilary size variation measures (perhaps the standardization step at the data reprocessing stage reduced its effect). More work could be done to refine the setup and understand the trade-offs between the fidelity of the equipment, environmental setup, and the robustness of results. 

!!!!Results the data for pupil is explored in a comapgnion paper and shown to best the others... we could recognize high task load low task load onset!!!

\subsection{Future work}
There are a number of investigative ways to approach the dataset, and we discuss three here (response filtering, ground truth calibration, and modeling stress source and timecourse). The first is with regards to learning temporal models, which include transitions between cognitive states. In the fields of cognitive science and psychophysiology, task-evoked responses are used to make fine-grained measurements on cognitive load, whereas human-computer interaction researchers generally use aggregated measures over a period of time to obtain more coarse and robust measurements~\cite{klingner2008}. By using overlapping  data averaging windows and statistical features we have merged the two approaches. Task load change modelling would certainly be improved by include the transition states from low to high task load discarded for the analysis in this paper. Temporal models based on task load change instead of static task load would speed and futher improve robustness of our already good prediction based on a sequence of observed physiological measures.  

A second approach brings into question the ground truth of the cognitive load experienced by the users. One option would be to use the recorded reaction times, or some composite of the different task performance measures, as a high resolution measure of ground truth. These could be used to learn regression models on the psychophysiological signals. The idea here is that if we can predict reaction times, we can estimate at a fine-grained level the cognitive load being experienced by the user. Another option would be to use the pupil dilation measure as a ground truth.

A third approach is based on differentiating between cognitive load and short-term acute stress. While there are no doubts that they are related, its not clear how to measure them independently, and how one confounds the other. There is good consensus in the literature regarding conditions where stress is likely to arise --- failure at a task, together with feelings of lack of control, in situations where participants are evaluated by others~\cite{conway2013}. We might thus hypothesize that stress is an affect, which ebbs and flows at a slower pace than cognitive load which fluctuates more rapidly, reflecting the stages of mental processing. It is possible that some physiological signals are better indicators of stress than cognitive load, while some are sensitive to both. A first step would be to contrast the signals in the baseline condition with the experimental conditions. Signals with longer drag times might be more reflective of stress. A related step would be to correlate the different measures, and explore how to effectively align them to get a more complete picture of a user's mental and affective state. 

\section{Conclusion}
This paper presents a data set showing cognitive load changes across task load changes.  The tasks were created to be difficult enough to elicit significant errors when a secondsary task was added for high task load.   The measurments included  Electrodermal Activity,MechanoMygram Electro, Skin temperatures, ElectroCardioGram , Photoplethysmoragraph, ImpedenceCariograph,  Respiration, gaze, and pupil dilation. Apart from the eye-based signals, which were sampled at 30 Hz, the rest of the signals were sampled at 2000 Hz. 
Deriviative signals of Instantaneous heartrate, Pulse Transit Time and Difference in Skin temperature are also calculated in the dataset.
!!! below update with your new experimental result!!
The burden is on the technology to determine the appropriate time to engage the user. One approach to do this is to gauge the cognitive load of the user using psychophysiological signals. While prior work has made progress in identifying such measures, lots of work still needs to be done to understand how to track rapid fluctuations in cognitive load in real time, especially in multitasking scenarios. To address this, we created a dataset of physiological measures, collected from participants in a multitasking study where they attended and responded to notifications while driving. We also collected a number of corresponding performance measures on both tasks. As an initial undertaking, we built models using statistical features extracted from pupil dilation measures. We showed that the resulting population model is capable of identifying the tasks the user is engaged in with an ROC AUC score of 0.85.

This paper presents a database eliciting cognitive function change accross high and low task load comparing 11 different physiological measures of cognitive function.  The tasks were chosen to elicit enough work to cause all subjects to make significant errors in the highest load conditions. Our initial findings show new valuable metrics for temporal change of cognitive load. We hope these to inspire other reserachers to use our dataset and to produce more such comparatie datasets as well.  


Our initial findings from this dataset show presented in a companinon paper []show that pupilary size variation was the only reliable measure for task load traccking that could track task loads in the near realtime 5 second range.  Indeed, our experminatation indicates that while modelling task load change might improve our anaysisis and bring the variation specificity down to 3 seconds beyond that is likely below the physiolotical response time of we could find.  
Initial exploration also found that pupilary dialation measures could be found and measured inside a car in daylight conditions.  The work also includes a new dataset in which machine learning model  on pupilary dialation was able to track task load change with better than 80 percent reliability.

This paper then presents that pupilary dialiation measures can be taken cheeply and in the wild to directly measure task load level in complex and coordinated tasks accross sensory motor verbal accuuision and cognitive task requests.
We see cognitive load assement as a rich area that could greatly improve human performance in many ways.  By measureing task load's effect we can modulate the difficulty of tasks we attempt.  By measuing task load effect computers can know when to reduce stimulation notification task requrest of any kind.  We hope this work inspires others to develop task load sensitive tests and user  experiences as well.



